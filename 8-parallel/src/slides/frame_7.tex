%----------------------------------------------------------------------------------------
%	SLIDE 7.
%----------------------------------------------------------------------------------------
\begin{frame}
\frametitle{Race condition}

\begin{block}{Párhuzamosítás a gyakorlatban}
	Több eszköz is nagyban megkönnyíti már párhuzamos kódok programozását:
	\begin{itemize}
		\item Python: \texttt{threading}, \texttt{multiprocessing} stb.
		\item C++17: Standard könyvtárak
		\item C/C++/Fortran: \texttt{OpenMP}, \texttt{CUDA}, stb.
	\end{itemize}
\end{block}

\uncover<2->{
\begin{alertblock}{Megoldandó problémák}
	\begin{itemize}
		\item Rengeteg lenne a fenti eszközök nélkül, de még így is vannak...
		\item Legfontosabb talán a \q{race condition}
		\begin{itemize}
			\item Két szál azonos memóriaterülethez akar egyszerre hozzábabrálni
			\item Egyik tipikus esete a rettegett \q{undefined behaviour}-nek
			\item Kritikus hiba, ami teljesen el tudja rontani bármilyen szoftver működését
		\end{itemize}
	\end{itemize}
\end{alertblock}
}

\end{frame}