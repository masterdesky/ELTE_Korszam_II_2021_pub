%----------------------------------------------------------------------------------------
%	SLIDE 10.
%----------------------------------------------------------------------------------------
\begin{frame}
\frametitle{Amdahl-törvénye}

\onslide<1->{
\begin{exampleblock}{Motiváció}
	\textbf{Mennyi időt nyerhetünk valós esetekben a párhuzamosítással?}
\end{exampleblock}
}

\only<2-3>{
\begin{itemize}
	\item<2-3> Egy program futásideje kifejezhető az alábbi módon:
	\begin{block}{}
		\begin{equation*}
			T = T * S + T * P
		\end{equation*}
	\end{block}
	ahol $S + P = 1$ és melyek rendre egy algoritmus kizárólag sorosan futtatható, valamint párhuzamosítható részeinek arányát jelöli.
	\item<3> Ha a párhuzamosítható (P) részt több (N) szálra osztjuk szét, akkor a program futásideje lecsökken:
	\begin{block}{}
		\begin{equation*}
			T_{új}
			=
			T * S + T * \frac{P}{N}
		\end{equation*}
	\end{block}
\end{itemize}
}

\onslide<4-5>{
\begin{itemize}
	\item<4-5> A felgyorsulás (Q) így felírható az alábbi módon:
	\begin{block}{}
		\begin{equation*}
			Q \left( N \right)
			=
			\frac{T}{T_{új}}
			=
			\frac{\cancel{T}}{\cancel{T} * S + \cancel{T} * \frac{P}{N}}
			=
			\frac{1}{S + \frac{P}{N}}
		\end{equation*}
	\end{block}
	\item<5> Tovább egyszerűsítve felírhatóvá válik az Amdahl-törvény megszokott alakja:
	\begin{block}{}
		\begin{equation*}
			Q \left( S, N \right)
			=
			\frac{1}{S + \frac{P}{N}}
			=
			\frac{1}{S + \frac{1-S}{N}},
		\end{equation*}
	\end{block}
	mely megadja, hogy a felgyorsulás kizárólag a sorosan futtatandó részek arányától és a szálak számától függ.
\end{itemize}
}

\end{frame}